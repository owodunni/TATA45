\documentclass[a4paper,12pt]{article}

\title{TATA45 Komplex analys}
\author{Alexander O Poole}
%% Definitioner för vågfysikrapporten-dokument

%% Text-kodning, språk samt PS-font
\usepackage[utf8]{inputenc}
\usepackage[T1]{fontenc}
\usepackage{ae,aecompl}
\usepackage{listings}
% % bitmap-grafik
\usepackage{graphicx}
% % matematik
\usepackage{mathtools}
\usepackage{latexsym}
\usepackage{graphicx}

%% Paragrafformat
\setlength{\parindent}{0pt}
\setlength{\parskip}{1ex plus 0.5ex minus 0.2ex}

%% Format för datum
\newcommand{\twodigit}[1]{\ifthenelse{#1<10}{0}{}{#1}}
\newcommand{\dagensdatum}{
\number\year-\twodigit{\number\month}-\twodigit{\number\day}}

%% Sidhuvud och sidfot
\makeatletter
\let\runauthor\@author
\let\runtitle\@title
\makeatother
\usepackage{fancyhdr}
\pagestyle{fancy}
\lhead{\runtitle}
\rhead{\runauthor}
\lfoot{alepo020@student.liu.se}
\cfoot{{\ } \\ \thepage}


%%Gray box
\usepackage{mdframed}

%% Absolut belop tecken
\DeclarePairedDelimiter\abs{\lvert}{\rvert}%
\DeclarePairedDelimiter\norm{\lVert}{\rVert}%

% Swap the definition of \abs* and \norm*, so that \abs
% and \norm resizes the size of the brackets, and the 
% starred version does not.
\makeatletter
\let\oldabs\abs
\def\abs{\@ifstar{\oldabs}{\oldabs*}}
%
\let\oldnorm\norm
\def\norm{\@ifstar{\oldnorm}{\oldnorm*}}
\makeatother

\newcommand*{\Value}{\frac{1}{2}x^2}%

  %%EX
  %\[\abs{\Value}  \quad \norm{\Value}  \qquad\text{non-starred}  \]
  %\[\abs*{\Value} \quad \norm*{\Value} \qquad\text{starred}\qquad\]


%%Dokumentets början
\begin{document}
\maketitle
\thispagestyle{empty}
\clearpage

\thispagestyle{empty}
\pagenumbering{roman}

\section*{Inledning}
Detta är en sammanfattning på kursen TATA45 komplex analys som ges på Linköping universitet. 
Sammanfattningen syftar till att med mina egna ord förklara kurs kompendiumet skrivet av Lars Alexandersson. 
Styckena i sammanfattningen kommer följa de i kurskompendiumet både med namn och numrering. 

\thispagestyle{empty}
\tableofcontents

\clearpage

\pagenumbering{arabic}

%#################################SAMMANFATTNING BÖRJAR##########################
\section{Tal, mängder och funktioner}
\subsection{Komplexa tal}
\begin{itemize}

Bra saker att ha koll på:

\item Ett tal som tillhör den komplexa talmängden skivs $z \in \mathbb{C}$ 

\item Konjugatet betäcknas $\bar{z}$ och innebär att det komplexa talet är speglat i realaxel dvs vinkeln byter täcken.

\item Lite konjugat räkneregler $\abs{z]^2=z\bar{z}$, $\bar{z_1 z_2}=\bar{z_1}\bar{z_2}$, 
$\bar{\frac{z_1}{z_2}}=\frac{\bar{z_1}}{\bar{z_2}}$ och $\bar{z_1 + z_2} = \bar{z_1} + \bar{z_2}$
\end{itemize}

\subsubsection{Triangelolikheten}
\begin{itemize}
\item $\abs{z_1+z_2}\leq \abs{z_1}+\abs{z_2} (Triangelolikheten)
Likhet gäller då $z_1$ och $z_2$, som vektorer är parallella dvs $\hat{z_1} \times$ \hat{z_2} = 0$.
\item $\abs{z_1 + z_2}\geq \abs{\abs{z_1} - \abs{z_2}}$ (omvänd Triangelolikheten)
Likhet gäller då $z_1$ och $z_2$, som vektorer är parallella och motsatt riktade.
\end{itemize}

%#################################1.2##########################
\

\end{document}


