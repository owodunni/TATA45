\documentclass[a4paper,12pt]{article}

\title{TATA45 Komplex analys}
\author{Alexander O Poole}
\input{style}

%%Dokumentets början
\begin{document}
\maketitle
\thispagestyle{empty}
\clearpage

\thispagestyle{empty}
\pagenumbering{roman}

\section*{Inledning}
Detta är en sammanfattning på kursen TATA45 komplex analys som ges på Linköping universitet. 
Sammanfattningen syftar till att med mina egna ord förklara kurs kompendiumet skrivet av Lars Alexandersson. 
Styckena i sammanfattningen kommer följa de i kurskompendiumet både med namn och numrering. 

\thispagestyle{empty}
\tableofcontents

\clearpage

\pagenumbering{arabic}

%#################################SAMMANFATTNING BÖRJAR##########################
\section{Tal, mängder och funktioner}
\subsection{Komplexa tal}

Bra saker att ha koll på:
\begin{itemize}

\item Ett tal som tillhör den komplexa talmängden skivs $z \in \mathbb{C}$ 

\item Konjugatet betäcknas $\bar{z}$ och innebär att det komplexa talet är speglat i realaxel dvs vinkeln byter täcken.

\item Lite konjugat räkneregler: 

\item $\abs{z]^2=z\bar{z}$, $\bar{z_1 z_2}=\bar{z_1}\bar{z_2}$, $\bar{\frac{z_1}{z_2}}=\frac{\bar{z_1}}{\bar{z_2}}$ och $\bar{z_1 + z_2} = \bar{z_1} + \bar{z_2}$


\end{itemize}

\subsubsection{Triangelolikheten}
\begin{itemize}
\item $\abs{z_1+z_2}\leq \abs{z_1}+\abs{z_2} (Triangelolikheten)
Likhet gäller då $z_1$ och $z_2$, som vektorer är parallella dvs $\hat{z_1} \times$ \hat{z_2} = 0$.
\item $\abs{z_1 + z_2}\geq \abs{\abs{z_1} - \abs{z_2}}$ (omvänd Triangelolikheten)
Likhet gäller då $z_1$ och $z_2$, som vektorer är parallella och motsatt riktade.
\end{itemize}

%#################################1.2##########################
\

\end{document}


