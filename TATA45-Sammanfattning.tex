\documentclass[a4paper,12pt]{article}

\title{Mekanik I, kapitel 1}
\author{Alexander O Poole}
\input{../style}

%%Dokumentets början
\begin{document}

\section{Punkters Kinematik}

OBS! Ska man vara korrekt då man pratar om initsialramar så ska man betäckna vilken
referensram lägesvektorn $r$ tillhör d.v.s. $r_r$. Detta struntar jag dock i då det
inte behövs på tentan. Vill man göra det ändå så set det ut på följande vis 
$r = r_r$, $v = v_r$ och $a = a_r$.

\begin{itemize}
  \item Referensram - stel odeformerbar kropp, man kollar ofta hur något rör sig i 
  förhållande till en referensram.I en given referensram $r$ kan man införa ett 
  kordinatsystem med basvektorerna $\hat{e}^r_i$ kan vi beskriva lägesvektorn 
  $r_r$ för en punkt i rörelse.

\end{itemize}

\begin{equation}
 r = \sum\limits_{i=1}^n r\hat{e}^r_i 
\end{equation}

\begin{equation}
v = \frac{dr}{dt} = \sum\limits_{i=1}^n \dot{r}\hat{e}^r_i 
\end{equation}

\begin{equation}
a = \frac{d^2r}{dt^2} = \sum\limits_{i=1}^n \ddot{r}\hat{e}^r_i 
\end{equation}


\subsection{Plana kurvors geometri}

  En kurva kan alltid parametriseras med avsende på en skalär dvs $u(t)$ 
  är entydligt bestämt.
  
\begin{equation}
 \hat{t} = \frac{dr}{ds}
 \label{t}
\end{equation}

$\abs{\hat{t}}=1$ vilket kommer visa sig väldigt smidigt senare.

\begin{equation}
\hat{n} = \frac{\frac{d\hat{t}}{ds}}{\abs{\frac{d\hat{t}}{ds}}} 
\label{n}
\end{equation}

 $\hat{n}*\hat{y} = 0 \text{ dvs } \hat{n} \text{ och } \hat{t} \text{ är ortogonala.}$
$\hat{t}$ pekar alltid i rikningen som punkten rör sig och $\hat{n}$
pekar alltid mot kröknings centrum i eventuell cirkelbana som punkten rör sig i.

krökningen för en godtycklig kurva efter sträckan s:

\begin{equation}
 k(s) = \abs{\frac{d\hat{t}}{ds}} 
 \label{k}
\end{equation}

krökning för en cirkel blir:

\begin{equation}
K = \frac{1}{R}   
\end{equation}

krökningen som en godtycklig funktion av båglängden $u$ betäcknas:

\begin{equation}
K(u) = \frac{\abs{\frac{dr}{du}\times\frac{d^2r}{du^2}}}{\abs{\frac{dr}{du}}^3} 
\end{equation}

Ett vanligt val av parameter är tiden t varav utrycket blir:

\begin{equation}
K(u) = \frac{\abs{v\timesa}}{\abs{v}^3} 
\end{equation}

\subsubsection{Oskulerande cirkel och krökningsradie}

krökningsradie :
\begin{equation}
\rho = \frac{1}{K}
\end{equation}

krökningscentrum:
\begin{equation}
r_c = r + \rho\hat{n}
\end{equation}

\subsection{Plan kinematik i naturliga basen}

\begin{equation}
v = \frac{dr}{dt}=\frac{dr}{ds}\frac{ds}{dt}=\frac{dr}{ds}\dot{s}
\end{equation}

(\ref{t}) ger:

\begin{equation}
v= \dot{s}\hat{t}=v\hat{t}
\end{equation}

OBS! \hat{t} kan defineras positiv i godtycklig riktning.

Farten $\abs{v}=\abs{\dot{s}}$ detta då \abs{\hat{t}=1}

\begin{equation}
a = \dot{v} = \ddot{s}\hat{t} +  \dot{s}\dot{\hat{t}} 
  = \ddot{s}\hat{t} +  \dot{s}\frac{d\hat{t}}{ds}\frac{ds}{dt}
\end{equation}

Detta tillsamans med (\ref{n}) och (\ref{k}) ger:

\begin{equation}
a = \ddot{s}\hat{t}+\frac{\dot{s}^2}{\rho}\hat{n}
\end{equation}

Om kurvan är en affin funktion blir krökningsradien oändligt stor dvs punkten 
har en rätlinjig rörelse:

\begin{equation}
a = \ddot{s}\hat{t}
\end{equation}


\subsubsection{Cirkelrörelse}

För punkter som rör sig längs en cirkulär bana med radien $R$ och läget $\theta$ 
gäller följande rörelse funktioner:

\begin{equation}
v = R\dot{\theta}\hat{t}
\end{equation}

\begin{equation}
a = R\ddot{\theta}\hat{t} + R\dot{\theta}^2\hat{n}
\end{equation}

\subsubsection{Varignons sats}

Vill man uttrycka sambandet mellan hastighet och acceleration utan att blanda in tid så kan 
sträcka eller hastighers förändring annvändas istället. Detta ger varigons sats:

\begin{equation}
\int a_t ds = \int vdv
\end{equation}

\subsection{Plan kinematik i katesiska koordinater}

Vill man annvända kartesiska koordinater för att täkna rörelse ekvationerna så får man 
följande rörelseekvationer:

\begin{equation}
r = x\hat{x} + y\hat{y}
\end{equation}

\begin{equation}
r = \dot{x}\hat{x} + \dot{y}\hat{y}
\end{equation}

\begin{equation}
r = \ddot{x}\hat{x} + \ddot{y}\hat{y}
\end{equation}

Varignons sats:
\begin{equation}
\ddot{x}dx = \dot{x}d\dot{x}
\end{equation}

\begin{equation}
\ddot{y}dy = \dot{y}d\dot{y}
\end{equation}

\subsection{Plan kinematik i polära koordinater}
I polära koordinater annvänds basvektorerna 
$\hat{\theta}$ och $\hat{r}$ för att bestämma en punkts läge. 
$r$ är längden av lägesvekron $\hat{r}$ och $\theta$ är vinkeln 
ellan en fixt linje i referensramen och lägesvektorn $\bar{r}$. 
$\hat{\theta} \times \hat{\r}= 0$ dvs $\hat{\theta}$  är 
ortogonal mot $\hat{\r}$ och riktad på ett sådant sätt att
$\hat{\theta}$ pekar mot växande värden på $\theta$.
Detta ger oss följande rörelse ekvationer:

\begin{equation}
\bar{r}=r\hat{r}
\end{equation}

\begin{equation}
v = \dot{r}\hat{r} + r\dot{\theta}\hat{\theta}
\end{equation}

\begin{equation}
a = (\ddot{r}-r\dot{\theta}^2)\hat{r}+(r\ddot{\theta}+2\dot{r}\dot{\theta})\hat{\theta}
\end{equation}

\subsubsection{Cirkelrörelse}

Om R är konstant dvs om vi rör oss längs en cirkel blir ekvationerna:

\begin{equation}
v = R\dot{\theta}\hat{\theta}
\end{equation}

\begin{equation}
a = R\ddot{\theta}\hat{\theta}-R\dot{\theta}^2\hat{r}
\end{equation}


\subsection{ordlista}

\begin{itemize}

  \item  Skalär - reala tal

  \item  krökning - $k(s)$ anger hur $\hat{t}$ ändras längs med kurvan

  \item  Oskulerande cirkel -  Om man på en given kurva skulle tangera 
         en punkt med en cirkel med samma krökning i punkten som kurvan.
         Så har man taggit fram den oskulerande cirkeln till punkten.

  \item Affine funktion - en linjär funktion adderad med en konstant

\end{itemize}

\end{document}


